\documentclass[a4paper,11pt]{jsarticle}


% 数式
\usepackage{amsmath,amsfonts}
\usepackage{bm}
% 画像
\usepackage[dvipdfmx]{graphicx}


\begin{document}

\title{デジタル映像処理\\第6回課題2}
\author{K19092\\福田遥生}
\date{\today}
\maketitle

\section{はじめに}


本レポートでは,図1に示した画像に対し水色の円のみの輪郭を抽出する方法を述べる.

\begin{figure}[h]
  \centering
  \includegraphics[height=3cm]{image/kadai.jpg}
  \caption{元画像}
\end{figure}

\section{処理手順および処理結果}
\begin{figure}[h]
  \begin{minipage}{0.5\hsize}
    \begin{center}
      \includegraphics[height=3cm]{image/bin1.png}
      \caption{二値画像}
    \end{center}
  \end{minipage}
  \begin{minipage}{0.5\hsize}
    \begin{center}
      \includegraphics[height=3cm]{image/bin2.png}
      \caption{膨張処理を2回おこなった図2}
    \end{center}
  \end{minipage}
\end{figure}
\begin{figure}[h]
  \begin{minipage}{0.5\hsize}
    \begin{center}
      \includegraphics[height=3cm]{image/bin3.png}
      \caption{収縮処理を10回行った図3}
    \end{center}
  \end{minipage}
  \begin{minipage}{0.5\hsize}
    \begin{center}
      \includegraphics[height=3cm]{image/bin4.png}
      \caption{膨張処理を2回おこなった二値画像}
    \end{center}
  \end{minipage}
\end{figure}

まず,輪郭抽出のために,二値化した画像に対して膨張処理を2回おこなった.

次に,その画像に対して収縮処理を10回おこなった.図形の外にあるノイズは8回目の収縮処理で削除できたが,膨張処理をおこなったことで画像中央の円2つが繋がってしまったため収縮処理を10回おこなっている.
その後,膨張処理を2回おこなうことで各図形のサイズを元のサイズに戻している.

最後に,各円の円周と面積を表1に示す.


\begin{figure}[h]
  \begin{minipage}{0.5\hsize}
    \begin{center}
      \includegraphics[height=3cm]{image/dist.png}
      \caption{出力画像}

    \end{center}
  \end{minipage}
  \begin{minipage}{0.5\hsize}
    \begin{center}
      \makeatletter
      \def\@captype{table}
      \makeatother
      \caption{各円の円周と面積}
      \begin{tabular}{|l|c|r|}
        \hline
        円  & 円周   & 面積     \\
        \hline
        円1 & 469.59 & 15720.00 \\
        \hline
        円2 & 251.52 & 4522.50  \\
        \hline
        円3 & 318.88 & 7238.00  \\
        \hline
        円4 & 283.08 & 5733.00  \\
        \hline
        円5 & 252.94 & 4547.00  \\
        \hline
      \end{tabular}
    \end{center}
  \end{minipage}
\end{figure}




\end{document}